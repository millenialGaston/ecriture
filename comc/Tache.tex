 \section*{Tache}
 nuages clairs une précision obscure des arbres valsent un chien court un
 chien aboi le chat l’attaque Jolie roule Jolie chante; un vélo vole des portes
 claquent des cloches sonnent des casiers qui grincent mordillent les doigts
 des lumières blanches qui tachent l’endroit ; c’est long c’est long ça finit
 pu de traverser les halles du  CEGEP de Rimouski Jolie y passe de
 longues journées à écouter des quadratiques des Habsbourg des martyrs brûlant
 dans les feux de joie d’Iroquois des électrons de valence des normes éthiques
 des philosophes obvious, tous se racontent en chœur, grande clameur d’un
 savoir qui semble parfois ma foi bien utile ou juste captivant. mais non la
 majorité du temps à vrai dire l’impression de se faire prendre pour des cons.
 Jolie est rousse mais ses cheveux paraissent d’un blond terne sous les néons.
 Les cheveux juste ras des épaules, bien droits lui cachent la moitié du
 visage, les yeux disparaissent en sourire lorsqu’elle voit des gens qu’elle
 aime, le reste du temps le regard froid glisse caché elle embusque elle traque
 la vermine on lui a apprit dans le dos ça se fait pas on parle en bien et on
 descend pas on ne mine pas pour se remonter tout le monde finit juste par se
 caler c’est évident mais d’autres rush comprennent pas comme Naomi la bitch
 qui se trouve toujours un soufre douleur ou l’autre bellâtre gossant qui lui
 jette tous les jours une un commentaire stupide, Jolie ne lui en veut pas. Il
 apprendra, à ses dépens, il faut apprendre un jour à se faire aimer autrement
 qu’en narguant et bousculant, s’pas grave on s’en fout de ces imbéciles. dans
 la vie il y mieux de toute façon, par exemple Zoé, qu’elle pensait absente
 aujourd’hui parce que malade du fond du couloir elle marche vers son casier.

Ce qui est bien avec Zoé c’est quelle partage tout, pas juste les cigarettes
les pointes de pizz ou les cartes pokémon lorsqu’elles étaient plus jeune
(sorry les billes c’était en France dans les années 40) mais l’humour aussi.
Elle apportait un cynisme jovial dans la journée scolaire maussade. Les mots
pour se moquer les yeux pour dire que c’est pas méchant.  Une tête sur les
épaules, on se dirait qu’elle avait surement vécu quelque chose de très triste
plus jeune à être aussi mature. Les fatiguants les gossants s’approchaient pas
trop d’elle. Zoé s’accote sur le casier de Julie et pousse un long soupir qui
veut plus dire grand-chose, c’est des ados après tout, la tête de biais pour
laisser tomber sa pluie de cheveux noirs charbons mordillés jusqu'à ses pouces
bien entortillés autour des sangles de son sac à dos style surplus d'armée


- Zo t’as tu fait le devoir d’Anglais?  \\
-[...]\\
-!!!\\


Zoé range ses cahiers de classe pour l’avant midi dans son sac à dos \\
Elle claque la porte et commence a marcher.\\
- En-voèye on va être en retard encore pour le cours de gym, moi je gosse pas
avec une baguette de badminton une minute de plus qu’il faut; le Jocelyn va
nous donner des exercices de plus\\\
- "Punition positive "\\
- "La fonction est d’améliorer "\\
- "On amène l’étudiant à aller à son propre potentiel "\\
- "C’est inspirant comme institution"\\
- "Fuck yé weird ce gars la"\\

Elles se sont rencontrées comme membres de la même chorale au début du
primaire. Bon elles n’étaient plus insérables comme avant depuis quelques
années déjà, l’adolescence l’identité, etc. Zoé est devenue plus rough sur
les bords, aimait provoquer et foutre la marde ; Jolie se voulait ouverte
d’esprit et jeune et aventureuse mais trouvait tout ceci un peu trop obvious et
juvénile, toute cette révolte, l’ex-centrisme manifeste.\\

Jolie et Zoé ont la chance de se faire créditer un cours pour leurs pratiques
de band; avec Mme. Ashkenazy comme prof et superviseuse, parce qu'évidemment
il faut s'assurer qu'elles \textit{méritent} leurs crédits. 
Mme ashkenaz est une musicienne tchèque d’une
quarantaine d’année, toujours habillée de corduroy. Aigre mais sympathique,
rigoureuse mais enjouée tout de même. Jolie rejoint Zoé dans la salle de
musique son enveloppe de guitare à l’épaule. Elles vont s’asseoir à l’une des
tables et préparent leur gear ; branchent les fils, allument les amplis,
ajustent le tone. Jolie grattouille des cordes en forme d’accords lorsqu’elle
chante mais c’est Zoé qui supporte vraiment la fondation harmonique à la
guitare et saupoudre le tout de fioritures mélodiques. Jolie chante les
chansons qu’elle écrit, retravaille et compresse depuis maintenant quelques
années Elles jamment un peu pour se détendre et se délier les doigts. Mme
Ashkenazy se réchauffe à la batterie, effectue quelques manœuvres, des exercices
techniques de coordination et d’étirements.
Elle frappe fort : TCHAK TCHAK CRAK CHOMP TCHAK.
Ça fait un vacarme mais la salle de cours est en fait dans une rallonge du
CEGEP, un peu en retrait et isolée – et froide et mal foutue – c’est
d’ailleurs pour ça que l’administration l’a proposée (on est pas imbécile aussi
bien s’arranger pour que le moins de monde soit dérangé, toute façon personne
n’en veut de ce local plein asbestoses et humide) proposée à Mme Ashkenazy pour
l’implémentation du nouveau cours à option : “création d’un band sous la
supervision d’un professeur”.

Mme Ashkenazy n’est pas une réactionnaire, elle accueille autant le folk jazz
que le punk progressiste, les chansons ont parfois un certain air de scandale
et pas de souci. Mais on ne perd pas son temps, lorsqu’on arrive à la pratique,
on est prêt on a fait les lectures ; nouvelles charts, pages de manuels
techniques, essais sur l’art, etc. Le band “vieux techs” commence à être bien
rodé et les membres n’ont qu’à s’échanger quelques brefs mots, un signe ou deux
et elles commençaient la pratique. \\

Jolie et Zoé, un café à la main sur un banc de parc qui fait face au fleuve
proche du bas de la ville. Agathe arrive, la grande mince, chef de l'équipe
de volleyball, mais elle fait ça pour avoir sa bourse pour décrisser du coin 
au plus sacrant tout le monde le sait bien, et un peu pour sa mère, qui 
l'accompagnait à chaque pratique en char depuis qu'elle avait 8 ans. Une punk
en pastel, elle vit mollement sa crise d'adolescence, c'est plus une crise
de condescendance. mais très raffinné, seules les personnes qui passent 
quelques années avec elles peuvent s'apercevoir de l'ironie dans sa voix
lorsqu'elle répond avec enthousiasme  aux directives d'un professeur gossant
au sourire douteux où à la directrice, cette dernière très fière de son
championnat de volleyball.\\[1ex]

-- salut les girlz\\
-- allo Gate, ça a duré plus long que prévu votre pratique\\
-- Ouais ostie parle moi en pas, jta boute, depuis qu'on a gagné
l'année passé tout le monde est sur notre cas, le mien en particulier\\
-- Combien déjà McGill te donne pour rentrer sur leur team \\
-- Ben ils me payent mes frais de scolarité\\
-- Ouais mais le CEGEP c'est pas genre 75\$\\
-- Non mais à McGill, et je suis sûre de rentrer en archi\\
-- Hmm makes more sense, mais quand même, moi faudrait me payer
cher en Ta\_bar\_nak pour jouer en tit shorts proche de toutes ces vieux caliss
qui disent venir pour "encourager la région". \\
-- ouache esti, veux tu ben pas me mettre des images dans la tête Zo,
anyways, \ldots \\[1ex]
Agathe se retourne pour faire face à la berge, se penche et ramasse une
vieille chaise de toile qu'elle laisse trainée là depuis quelques années,
parce que trois sur un banc, c'est malaisant, il faut que le milieu recule
ou que les deux extrémités se penchent, une vieille chaise de toile donc, 
verte forêt, après l'avoir légèrement tapotée et secoué pour enlever les débris 
de terre elle s'affale dedans et sort son six-pack




--tit shorts à part vous les êtes là depuis combien de temps vous
autres\\
-- genre\ldots, depuis 2heure trente à peu près, Jo à quelle
heure est-ce qu'on a fini notre pratique\\

Jolie n'entend pas l'interpellation de sa camarade, elle rêvasse en sirotant
son café depuis qu'Agathe est arrivée, même un peu avant pour être honnête, 
probablement 5 minutes après que Zoé ai commencé son rant sur la nullité
de la musique quebz qui passe à la radio. Elle a les sourcils légèrement
froncés, on dirait qu'elle regarde très loin mais en fait ses yeux sont perdus
dans la marée qui récède, les mains dans les poches, légèrement crispée, le 
vieux k-way qui protège de la brise, une rumination quelconque.
C'est aussi un lieu quelconque qu'elles ont adoptées pour se rencontrer,
tergiverser un peu entre cafés ou bières. Ce petit parc à l'aube de la berge 
fait partie des derniers efforts de la mairies pour rendre l'urbanisme quelque
peu plus moderne. Une piste cyclable longe le rivage proche du
centre-ville, à certains intervales, décorée de quelques bancs de parcs et
de grosses chaises et tables étranges en palletes de bois recyclés. Les bancs
de parc sont accouplés à des lampadaires aux lumières intelligentes aux ampoules
LED, c'est un jaune opaque et mat qui rayonne le soir et virovolte contre les
arêtes de l'eau et les épinettes parsemés derrière la piste cyclable pour
écorcher les vents un peu trop vivants. \\

Ce n'est pas rare que Jolie ait ces moments d'absence, on pourrait dire qu'elle
arbore un léger TDH si on ne distinguât pas la concentration et la capacité
de se faire une sieste de l'intérieur. 

--Jo?\\
--Aloo décroche?\\
--hmmm\ldots?\\
--Fak est-ce que tu viens au Bunker demain soir finalement?\\
--Je pense que oui, vous avez trouvé un lift?\\
--Gate prend le char de son frère\\
--Le pickup?\\
--Ouais\\
--Mais il a pa genre juste deux places, on était supposés ammener
Andrée aussi\\
--pas de stress, j'ai des petits coussins pour mettre dans la boîte, même
un petit cooler avec des bières pour la route\\
--Et ta mère est chill avec ça?\\
--Bof ma mère à partir de 21h elle est accotée sur le xanax et le ballon de
  rouge en train d'écouter Télé-Québec, elle remaquera même pas qu'on est
trop pour fitter dans le char. Pis toi Jo t'as récupéré tes affaires de camping
chez ta mère?\\
--Non mais ça va j'ai un sleeping bag et une bâche j'vais m'arranger.\\
--hmmm t'arranger enh? alors c'est qui que tu pensais te pogner?\\
--honnêtement, j'ai spotté un shed à quelques minutes de marche
la dernière fois donc si il y a pas de place dans ta tente, 
admettons que gate soit \textit{par hasard, pas} en train 
de se pogner Doménico 
--eille comment t'as entend\ldots
--peut-importe, donc à moins qu'il y ait pas de place dans la tente
je vais me prendre une petite marche et aller décanter la concoction
toute seule en mode boudhiste dans le shed sur le bord de l'eau\\
--avec ce qu'il va y avoir dans la potion magique je suis pas sur 
que tu veuilles vraiment t'égarer dans les bois comme une petite
Hansel
--Gretel\\
--Quoi?\\
--Gretel c'est la fille, Hansel c'est son frère\\
--Whatever\\
--\ldots\\
--\ldots\\
-- fak domenico enh?\\
--Si tu le dis à mon frère tu vas manger ta gaspacho chaude fille\\
--Wow\\
--$\backslash$  o/
\clearpage
