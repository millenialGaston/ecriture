\clearpage
\section*{océan}

L’institut des sciences de la mer de Rimouski est un centre de recherche
affilié à l’UQAR, on y étudie tout ce qui a lieu aux grandes étendues d’eau.
Bien entendu on pense surtout au golfe du Saint-Laurent. Les océanographes qui
y travaillent se déclinent en plusieurs profils ; géologues marins,
biologistes, on étudie la géophysique des courants et le plancton et ses effets
sur la faune.  C’est donc diversifié comme milieu, surtout depuis les dernières
initiatives du gouvernement qui ont pour but d’attirer les immigrants en
région. Paul Diez est chercheur en dynamique des courants thermos-salins.
Exposition sommaire du phénomène : l’eau chaude des tropiques se déplace vers
les pôles puis se refroidit, elle devient plus dense elle descend vers les
profondeurs, la salinité la rend plus lourde ; le plancher océanique est glacé
et salé. C’est à quelques milliers de profondeurs que la pression est assez
forte pour permettre à plus de sel de se dissoudre dans l’eau.\\

L’eau remigre par la suite vers l’équateur où elle se réchauffe et remonte,
l’agencement du tout produit les grands courants océaniques. En bas, dans l’eau
froide et noire ce pourrait être effrayant, avec ces poissons étranges tout
droit issus du Jurassique on dirait. Ces parcours de milliers de kilomètres
autour du globe fascinent Paul Diez, surtout la couche profonde de l’océan ;
l’abîme. Avec ces drôles de poissons, ils sont mignons après tout, et ils ne
veulent pas vraiment de mal à personne. Ils ont l’air plutôt paisibles ces
petits monstres laids.\\

Paul n'aime pas beaucoup les gens, à quelques exceptions près. Il s'imagine un 
Cousteau détennant plus de moyens techniques malgré le financement plutôt 
dérisoire que lui accorde le gouvernement ces derniers temps. Il a nommé son
bateau de recherche le Nordique, dans un espoir piqué d'arracher quelques
sympathies à ces philistins de la capitale qui s'épanchent encore en mélancholiede leur désormais disparue équipe de hockey qui portait ce nom.



Paul pilote de chez lui un petit sous marin télécommandé. Il se promène ainsi à
des kilomètres de profondeurs dans le confort de son bon fauteuil mou. Parfois
il va physiquement dans un plus gros seaexplorer avec des bons sièges et des
biscottes mais il coûte cher à l’université. L’administration voit toutes ses
promenades scientifiques d’un œil sceptique. Certains d’entre eux sont un peu
morons faut le dire.  \\
\clearpage
La mère de Julie habite à quelques dizaines de kilomètres de la ville. Sa fille
ne comprend pas encore très bien qu’est-ce qu’elle fait pour gagner sa vie au
fait. C’est un mélange bizarre de job, elle est boulangère à ses heures,
conseillère de ville à d’autres, on a eu ouï dire qu’elle a passée son barreau
autrefois pourtant elle passe plus de temps à contempler et nourrir ses chèvres
qu’a lire les journaux, si elle lit c’est de la poésie, un peu de Tchèque et du
français bien entendu mais aussi de l’américain et elle s’essaie récemment au
portugais ce qu’elle essaie de transmettre à sa fille. “T’aimes le jazz et la
samba, c’est beau la bossa, tu pourrais chanter des balades brésiliennes?”
\clearpage
