\section*{Cambré et Serein}

\setlength{\parindent}{0cm}
Messieurs Cambré et Serein sont des professeurs de renommée
international. Professeurs de quoi, on ne saurait le dire. Et c’est
justement à cela qu’ils doivent leur grand succès.\\

Une fine pluie a laissé une trace d'humeurs passées; une nostalgie
incertaine teintée d'arômes de câpres, câpres nuagés de fromage à la
crème autour d’un banc de parc de la place Émilie Gamelin. Cambré et
Serein prennent place. Ces deux hommes digèrent une ère future
bucolique idéalisée à travers la vitrine de leur vie antérieure. Dépit
et cynismes se sont installés en leur cœur. Cynisme cependant mitigé.\\

Messieurs les Professeurs Cambré et Serein allument chacun une
cigarette, l’air paisible. L’on peut sentir un fond de brume
d’automne, une lueur grisâtre et vaseuse. Cette vase a le pouvoir
d’abrier le présent de l’écoulement incessant du temps. \\

Pour l’instant le présent s’accomplit à coup de minutes
non-évènementielles, monochromes. La seule attente est cristallisée
dans la conicité parfaite du joint que
Prof. Serein est sur le point d’humecter de gestes précis qui
démontrent une l’habileté certaine. Prof. Cambré quant à lui est sur
le point de trouver une petite phrase sur laquelle poser la fondation
des quelques 30 minutes subséquentes de disage de marde quand soudain
un bruit retentit qui arrache le vide de la page blanche, un bon
sobriquet quoi. Un colambule.  Une prostituée connue du quartier au
nom flatteur de Cosette, (son proxénète Joey avait eu des aspirations
littéraires étant adolescent,) cette jeune Cosette vient de se faire
bitch-slap  à terre solide. Le genre de soufflet qui priorise les
répercussions sonores et psychologiques sur l’efficacité physique.\\

- Hoho mon cher collègue, avez-vous vu ce beignet. \\
- Ah oui Professeur, Joey est en forme aujourd’hui dites donc!\\

Cosette se lève à moitié mais ce fait crisser à terre par un soufflet
d'une violence doucement étouffée par le beau feutre pourpre du tissus
employé. Elle se relève sur les coudes et parcourt quelques mètres
avant de s'arrêter pour s'époumoner d'une façon éminemment désagréable
pour tous les partis concernés. Elle se trouve maintenant à une
quinzaine de mètres derrière le banc en question.\\
-Toute une confiture!,selon Maitre Cambré

L’aurore commence à percer
doucement, les écriteaux de néon scintillent du coin de l’immeuble de
4 étages d’en face. Le parc, petit écrin de verdure, commence à voir
les quatre rues qui l’enserrent comme une ceinture pas trop chaste se
réveiller. \\

Face au banc s’étire st-Catherine au coin de laquelle
s’allume le nom de l’enseigne de l’Archambault imbriqué dans quelque
motif.\\

Un étudiant de l’UQAM munit de sa caméra 16mm et de son coat de
cuir est sur ce même coin de rue à filmer une capote virevolter et
onduler lentement dans le vent. Appelons le CoatDeCuir, c’est
logique.\\

Pendant que ces grands esprits s'avancent dans une analyse de la
possibilité de vivre une vie éthique, la pute sale continue un peu à
ramper en poussant des gros râles, ils commencent d’ailleurs à
légèrement s’éreinter.\\

On peut voir un couple de jeunes
professionnels, 28 ans peut-être, qui marchent main dans la main,
quelque peu en retrait. Ils ont un rire candide et frais lorsqu’ils
s’échangent inside et clins d’œil. Une couche de réalité résumée dans\\

Cambré: \textit{Au travers de tout cela il y a une aura, les écritaux me transme
    tent l'image \\[1ex]
    sortie au musé un dimanche matin avec une jolie étudiante aux yeux en
    amande, journée étirée entre blagues douteuses, regards en coin et
    conversation profonde, s’éteignant en étreintes vigoureuses; étreintes
    qui gardent un sentimentalisme et une naïveté profonde malgré les
    claques sur le postérieure et les effusions éjaculatoire subséquentes;
    une aura virginale consacrée à travers le sexe sale romcomisé;
    cette journée vous a été vendue en concept mais non en substance. La
    substance s’est enfouit dans une trame narrative teintée d’impressions
    cinématographiques, vos deux globes vitreux eux n’ayant jamais su
    l’insérer dans une escale concrète du quotidien.}

\hspace{3em}-cynisme \\

\begin{comment}
On pourrait
croire le cynisme de ces deux personnages déshumanisant. M. Cambré et
M. Serein n’offrant aucune interjection aux soufflets de Joey. Bien
d’abord ils ont une relation plutôt complexe avec ce dernier, qui aide
sa communauté de mainte façon avec ses revenus de maquereau.   M.
Cambré est malheureusement atteint d’une douloureuse arthrite des
doigts. Non seulement cela lui empêche de se porter au secours de
quelque individu d’une manière physique mais de plus, étant pianiste
amateur de grand talents à ses heure, la seule façon qu’il peut
s’appliquer le baume d’une sonate de Béthov sur la gale d’angoisse
métaphysique qui l’assiège de jour en jour est de se faire une dose de
cheval de Dilaudid (DeelO) dont la seule source convenable est Joey.
M. Serein lui est atteint d’épisodes de neurasthénie particulièrement
sévères qui lui enlevant périodiquement l’usage de ses jambes le
confinant ainsi a son banc de parc jusqu’à ce que la brume morose qui
l’afflige finisse par passer.  Donc voyez-vous, avant de juger
l’homme, il faut

\hspace{5em} \textit{contexte} \clearpage
\end{comment}
\clearpage
