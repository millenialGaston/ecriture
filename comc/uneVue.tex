\clearpage
\section*{une Vue}
Jolie écrit des paroles de chanson dans son garage qui fait office de
chambre/tanière en regardant le fleuve qui se verse plus loin par la porte de
garage vitrée. Il a été aménagé pour elle il y a quelques années. C’est une
petite bâtisse de bois détachée de la maison par quelques dizaines de mètres.
Le terrain de M. Paul Diez est en pente à flanc de montagne, flanc de butte
pour être plus précis mais c’est assez ça fait que le soleil perce et l’on voit
ien la berge qui se reflète.  C’est probablement dans ces moments qu’elle est
le plus productive, de 7pm à 3 heures du matin environ. Elle a  soupé et peut
s’installer tranquillement dans le garage. Jolie s’en ai fait un nid avec un
grand tapis un vieux sofa et des disques qui traînent un peu épars. C’est
relaxant comme endroit, du Valium en pin blanc. La soirée est d’autant plus
productive si c’est l’été et une grosse pluie vient barboter sur l'eau au
loin.\\

Elle était un peu émèchée en rentrant du parc, quelques bières d'après-midi
avait suffit. Paul ne s'apercevait pas de ses choses là même s'il se levait de
son fauteuil; ce qu'il faisait de moins en moins. Jolie se fit donc une théière
de thé noir, n'aimant pas être trop vaseuese, elle rêvassait déjà bien assez.
Le garage est muni d'une mini cuisine de camper ainsi que d'une toilette, il
faisait donc office d'appartement temporaire avant de finalement pouvoir
s'eclipser, on ne sait où, ce n'est pas trop important, tant que ce soit
ailleurs. Le plus important, après le climat (svp plus habitable) serait la
distance. mais la distance c'est difficile, ça prend d'autres langues, des
avions, de l'énergie. Peut-être New-York, mais on se résignera bien sûr pour
Montréal, pas Québec en tout cas. Jolie sait que des ses camarades la majorité
opteront pour cette dernière lors du saut à l'université , on irait plus loin
en restant chez soi que d'aller là-bas. 




\clearpage
