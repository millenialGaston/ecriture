\clearpage
\section*{Nid}
Quatre années plus tard Jolie réside à Montréal, dans le quartier de la petite
italie. Trois colocataires, toutes gentilles, le grille pain est efficace, il y
a une petite galerie en avant avec un set de patio éclectique, des tas de
coussins et des chaises adirondaques. C’est le début de l’été, elle s’assoit
sur l’un des fauteuils, fait ses lectures en après-midi. Elle a apporté avec
elle dehors quelques volumes de poésie et des revues type national-geographic
avec des grandes photos de mammifères marins immenses et paisibles et des
chutes d’eau tropicales comme si c’était le monde qu’elle habite.  La rue
Casgrain lui fait face elle prend une pause pour s’étirer une heure ou deux
après s’être réveillée, boit un café et fait du people watching en mangeant une
courge spaghetti. Elle range un peu les coussins, taponne le tout, un bol de
salade au couscous traîne quelque part, une dernière bouchée, le soleil ne
devrait pas tarder à s’éteindre. Depuis quatre ou cinq mois c’est Cédric qui
visite, plus jeune de quelques années, il est mignon et gentil quelque peu naïf
et anxieux, mais il séduit avec ses yeux d’ailleurs, d’un peu plus loin.

Il débarque de son vélo, lui glisse un sourire, s’assied a terre, lui demande
de raconter sa journée. Il reste de la lumière ils en profitent. Le temps ça se
caresse ça se domestique, on lui donne des commandes avec des biscuits et du
chocolat les minutes grésillent comme un bruit blanc, le ciel délavé vieux
jeans. La chambre est à repeindre juste les bobettes à remettre il en met
partout il se tache et elle se fout de sa gueule il n’est pas doué. La pizza
est à terre Jolie aussi, proche de sa proie, assise en lotus, la bière aux
lèvres.  Ça finit dans le lit, même si l’odeur de peinture c’est pas génial
c’est l’été faut bien se gâter se faire du bien. Ils se promènent et mordillent
les draps, les draps volent Jolie chante. C’est simple et collant, ils
s’endorment, couchés en croix une tête sur le ventre de l’autre, des oreillers
qui traînent. Un peu de musique, ça se mélange au vent et au ronronnement du
fridge.\\

Elle a un soupir, lui un pet. Les deux rient, ils s’endorment.\\


Cédric est un peu pathétique lui laisse des poèmes écrits en coin de tables à
côté du matelas au sol. Elle dort un peu encore, c’est la sieste, ce soir elle
chante dans un bar. Ça la touche malgré tout ; elle en garde quelques un par la
suite, ils la suivent dans une petite boite en carton, par exemple :\\[1ex]
Avec tes taches de rousseur, poussières de feu\\
ça éclate tu es mon camion d’aube tu\\
verse dans le large une greffe de rayons\\
jette les murs pour des clairières\\
l’herbe haute l’air sec m’exfolie\\
le creux du sourire\\
‘ s’ouvre et on se berce hier s’arrête\\
demain commence après on verra\\
peut être\\
à petits pas\\
dors sans moi t’es bien\\
tu t-loves un peu dans les draps\\
d’une journée sans fin, ça s’étire \\
d’être de même, comme avars de paix \\
j’hallucine l’écrin je le sais\\
le vrai se condense pas\\
sur des brillants de douceur\\
Il faut que les vents fauchent de la scrape\\
l’amène dans les airs il faut\\
des noyaux pour que ça condense,\\
un grain de sel\\
une tache de poussière\\
tes taches de rousseur\\
