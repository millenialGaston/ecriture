 \section*{Tache}
 nuages clairs une précision obscure des arbres valsent un chien latexcourt un
 chien aboi le chat l’attaque Jolie roule Jolie chante; un vélo vole des portes
 claquent des cloches sonnent des casiers qui grincent mordillent les doigts
 des lumières blanches qui tachent l’endroit ; c’est long c’est long ça finit
 pu de traverser les halles de la polyvalente Rimouskienne Jolie y passe de
 longues journées à écouter des quadratiques des Habsbourg des martyrs brûlant
 dans les feux de joie d’Iroquois des électrons de valence des normes éthiques
 des philosophes obvious, tous se racontent en chœur, grande clameur d’un
 savoir qui semble parfois ma foi bien utile ou juste captivant. mais non la
 majorité du temps à vrai dire l’impression de se faire prendre pour des cons.
 Jolie est rousse mais ses cheveux paraissent d’un blond terne sous les néons.
 Les cheveux juste ras des épaules, bien droits lui cachent la moitié du
 visage, les yeux disparaissent en sourire lorsqu’elle voit des gens qu’elle
 aime, le reste du temps le regard froid glisse caché elle embusque elle traque
 la vermine on lui a apprit dans le dos ça se fait pas on parle en bien et on
 descend pas on ne mine pas pour se remonter tout le monde finit juste par se
 caler c’est évident mais d’autres rush comprennent pas comme Naomi la bitch
 qui se trouve toujours un soufre douleur ou l’autre bellâtre gossant qui lui
 jette tous les jours une un commentaire stupide, Jolie ne lui en veut pas. Il
 apprendra, à ses dépens, il faut apprendre un jour à se faire aimer autrement
 qu’en narguant et bousculant, s’pas grave on s’en fout de ces imbéciles. dans
 la vie il y mieux de toute façon, par exemple Zoé, qu’elle pensait absente
 aujourd’hui parce que malade du fond du couloir elle marche vers son casier.

Ce qui est bien avec Zoé c’est quelle partage tout, pas juste les cigarettes
les pointes de pizz ou les cartes pokémon lorsqu’elles étaient plus jeune
(sorry les billes c’était en France dans les années 40) mais l’humour aussi.
Elle apportait un cynisme jovial dans la journée scolaire maussade. Les mots
pour se moquer les yeux pour dire que c’est pas méchant.  Une tête sur les
épaules, on se dirait qu’elle avait surement vécu quelque chose de très triste
plus jeune à être aussi mature. Les fatiguants les gossants s’approchaient pas
trop d’elle. Zoé s’accote sur le casier de Julie et pousse un long soupir qui
veut plus dire grand-chose, c’est des ados après tout - Zo t’as tu fait le
devoir d’Anglais? Google Translate a pas encore été inventé, c’est de la marde;
faut toute chercher les mots un par un dans l’esti de dictionnaire
- ouais …\\\\
	comme tout le monde, t’as pas google l’année d’invention de Google 	Translate	? On doit toutes se tapé ça \\
- Esti j’ai hâte d’avoir un problème de drogue pour justifier des journées longues et vides de mêmes.\\
\\
Zoé range ses cahiers de classe pour l’avant midi dans son sac à dos \\
Elle claque la porte et commence a marcher.\\
- En-voèye on va être en retard encore pour le cours de gym, moi je gosse pas\\
avec une baguette de badminton une minute de plus qu’il faut; le Jocelyn va\\
nous donner des exercices de plus - « Punition positive »\\
- « La fonction est d’améliorer »\\
- « On amène l’étudiant à aller à son propre potentiel »\\
\\
- C’est inspirant comme institution\\
- Fuck yé weird ce gars la\\



Elles se sont rencontrées comme membres de la même chorale au début du
primaire. Bon elles n’étaient plus insérables comme avant depuis quelques
années déjà, l’adolescence l’identité, etc. Julie est devenue plus rough sur
les bords, aimait provoquer et foutre la marde ; Zoé se voulait ouverte
d’esprit et jeune et aventureuse mais trouvait tout ceci un peu trop obvious et
juvénile, toute cette révolte, l’ex-centrisme manifeste.
