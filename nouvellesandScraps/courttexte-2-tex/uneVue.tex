\clearpage
\section*{une Vue}
Jolie en pratique chante les mots qu’elle écrit souvent le soir la veille sur les
compositions originales de Mme Cymbales dans son garage qui fait office de
chambre/tanière en regardant le fleuve qui se verse plus loin par la porte de
garage vitrée. Il a été aménagé pour elle il y a quelques années. C’est une
petite bâtisse de bois détachée de la maison par quelques dizaines de mètres.
Le terrain de M. Paul Diez est en pente à flanc de montagne, flanc de butte
pour être plus précis mais c’est assez ça fait que le soleil perce et l’on voit
ien la berge qui se reflète. C’est probablement dans ces moments qu’elle est
le plus productive, de 7pm à 3 heures du matin environ. Elle a  soupé et peut
s’installer tranquillement dans le garage. Jolie s’en ai fait un nid avec un
grand tapis un vieux sofa et des disques qui traînent un peu épars. C’est
relaxant comme endroit, du Valium en pin blanc.  La soirée est d’autant plus
productive si c’est l’été et une grosse pluie vient barboter dans le fleuve
qu’elle regarde. 

La mère de Julie habite à quelques dizaines de kilomètres de la ville. Sa fille
ne comprend pas encore très bien qu’est-ce qu’elle fait pour gagner sa vie au
fait. C’est un mélange bizarre de job, elle est boulangère à ses heures,
conseillère de ville à d’autres, on a eu ouï dire qu’elle a passée son barreau
autrefois pourtant elle passe plus de temps à contempler et nourrir ses chèvres
qu’a lire les journaux, si elle lit c’est de la poésie, un peu de Tchèque et du
français bien entendu mais aussi de l’américain et elle s’essaie récemment au
portugais ce qu’elle essaie de transmettre à sa fille. “T’aimes le jazz et la
samba, c’est beau la bossa, tu pourrais chanter des balades brésiliennes?”
