\documentclass{article}

\usepackage{comment}
\usepackage[english]{isodate}
\usepackage{graphicx}
\usepackage{siunitx}
\usepackage{paracol}
\usepackage[utf8]{inputenc}
\usepackage{csquotes}
\usepackage{setspace}
\usepackage[bookmarks=true]{hyperref}
\usepackage{bookmark}
\usepackage{pdfpages}
%\includepdf[pages={1}]{myfile.pdf}
\renewcommand{\baselinestretch}{1.5}
%%For definitions:
\usepackage{amsthm}
\theoremstyle{plain}
\newtheorem{thm}{Theorem}[section] % reset theorem numbering for each chapter
\theoremstyle{definition}
\newtheorem{defn}[thm]{Definition}

\usepackage[margin=1in]{geometry}
\setlength{\parindent}{0pt}






\begin{document}
\section*{Intérieure lactée}
Mettez un film dans ma vie que j'en sorte.
C'est l'histoire d'un réveil constipé et ahurit suivit d'une journée inutile.
Les couleurs se coulent dans leur tiédeur ternes et les circonvolutions de mon
âme se complaisent en épithètes chialeux. Le café est trop lent et j'ai le temps
donc j'allitère, je pondère, il--c'est à dire le café--se déploie dans la tasse,
comme une routine de yogi au sourire imbécile, mielleux et perdu mais avec quelque
chose qui cloche derrière la paix intérieure lactée. La  méditation n'est pas
pour moi, je manque de flexibilité, bien loin du full-lotus même le demi-lotus
ne me convient pas. Et méditer sur une chaise, c'est con tout de même, c'est
contre le principe.\\

Dans ces cas, on me l'a toujours dit, on va faire du sport, une bonne course
dynamique pour se brasser les os --ou l'on fume. --L'on fume si on a trop
internalisé les codes des films nouvelle-vague, chose que j'ai réussit; à faire
malgré ma totale ignorance en ce qui concerne Jean-Luc Godart, la fuite du
cliché et de la tribus aboutissant toujours et inévitablement en cliché, Mais
tout-de-même, il faut meubler sa jeunesse.\\

Et je me trouve à un endroit où les meubles ne sont pas ce qui manque.  Ça
l'alterne entre le contemporain lisse, le canapé ancien-régime, la bay window
entre deux vases chinois, on a droit à du marbre, beaucoup de marbre, et un bois
disons, aventurons nous à dire japonais, sommes toutes, soyons clairs:
\emph{hétéroclite-nouveau-riche} est le terme qui convient à l'ensemble
architectural, même épanchons nous dans une caractérisation de l'atmosphère, des
gens, de la déco, du bâtiment et de la météo. Tous ces aspects ce soir dans ce
condo répondent au terme de \emph{hétéroclite-nouveau riche}. Ne nous résignons
pas à abandonner toute nomenclature esthétique tintée d'un  classisme
indéfendable, abandon par le biais d'une pudeur éro-conomique. Mais, Quoi-qu'on
en dise ça en jette. Le marbre les talons les grands verres, très grands verres
à vin, tout est brillant et cristallin, avec de légères notes complémentaires 
de soyeux et de velour-eux. Nous ne sommes plus ci-jeunes ((David est
maintenant un spécialiste des produits fixed-income s'est acheté un condo dans
lequel il est en train d'emménager avec Julie qui est toujours aussi empathique
et Jolie (nous nous sommes rencontrés par amis mutuels au cegep quand les
nouvelles rencontres étaient encore spontanées naïves et douces) ils s'aiment
c'est beau mais on sent qu'il y a un courant froid qui sinue autour de leurs
profils si bien dessinées et ça me réjouit, peut-être, tout ceci est confus et ça
ne se choisit pas les sentiments, pas envers Julie et d'envie face à la
situation de David, mais plutôt d' amertume face au bonheur d'autrui, quoique
disons le, soyons honnêtes,  Julie est très jolie) Jean lui est ingénieur et
fait le tour du monde, il sort d'où on sait pu trop, la Zambie, toujours la
Zambie et la Malaysie surtout d'où il revient avec ses histoires rocambolesques,
une légère barbe hirsute, de nouvelles normes culturelles et une nouvelle
personnalité qui vient se graffer sur ce qu'était Jean pré-nouveau voyage qui
change toujours mais toujours grand et blond et blanc, en fait tant qu'à y être
n'oublions pas Joe, ses lunettes rondes et son humour décapant, son charisme de
dents tachées, dents qui n'affectent pas son charisme car il peut se le
permettre avec ses cheveux gras et lisse, ses yeux sombres et son teint olive,
ses larges poignets ses yeux olives et son regard ombrageux, son je-men-foutisme
maintenant garni d'un concluant salaire à la radio de Radio-Canada), moi pour
faire rapide: pédant et dépendant financièrement, moi et eux. Donc on monte un
ascenseur au vieux-port un ascenseur qui fait zouuu tout en douceur avec un
cockpit comme si l'on voyageait dans un tube pas pneumatique mais amniotique.
Donc on monte dans ce tube et ça fait zouuu et on giggle entre quelques gorgées
partagées de vin blanc à la bouteille. Et l'on cogne entre deux simagrées
à cette grande porte lisse et pleine. Et l'on rentre dans cet appartement
mezzanine dont les deux étages donnent sur une immense fenêtre qui elle
même donne sur le centre ville illuminé et le fleuve qui s'allonge. 
Bien évidemment il y a du trap, un mobilier de jeunesse flétrie--disons
fin vingtaine à fin trentaine--riche, bon rien de dynastique mais tout de même,
en 2018, le mobilier d'une telle cohorte \emph{nécessite} le trap.
%
\footnote{Le trap est un style musical qui a ses origines dans le hip-hop du sud
	des états-unis. Il est marqué par de très rapides coups de snare en
	triplettes sur de larges basses lines qui ondulent sous le rythme de
	gros gras kick-drum. Le tout est garnit alors de \textit{mumble rap}, un
	style de rap où l'artiste déploie paresseusement ses rhymes, lorsqu'il y
	en a, avec l'accent d'un ivrogne sur la codéine, le rythme encore en
	triplettes: tatata-tatata-tatata-TA. Nous pourrions qualifier ce dernier
style d'une série de dactyles punchés à la fin par un anapeste moderne} Le condo
est situé au dernier étage d'un nouvel immeuble au vieux port de Montréal, les
planchers de granit peut-être.

%
Le matin ahurit se résorbe tranquillement. Nous avons fait la chose à faire,
après journée constipée on relâche l'appareil. On échanges quelques bières car
on est Samedi après tout et on se ramasse par quelque mécanisme obscur dans un
grand condo vitré au vieux-port de Montréal, entre deux galleries trop chères
qui vendent plus du design graphique commercial léché que de l'art, que l'on
se retrouve à gigglé avec des petits regards admiratifs en coin, malgré nous, en
fait malgré moi, car David et Julie sont habitués à l'endroit, pas précisément
celui-ci mais son essence, son zeitgeist, Jean est ben trop high pour avoir 
une quelconque appréciation esthétique soutenue qu'il se trémousse déjà en 
se faisant aller les bras vers la partie plus sombre de l'endroit où le dance
floor a été méticuleusement déposé, et Joe, Joe cherche déjà les verres et n'en 
a rien à foutre vraiment des bâtisses, il cherche des verres surtout pour se
chercher un verre parce que la bière ça fait pas la job et il a judicieusement
ammené un fiable 26oz de Jim Bean\footnote{Le Jim Bean est un whiskey, un
bourbon pour être plus précis, connu comme étant typiffiant de
l'amérique avec un gros r sale, d'une toxicité masculine, avec sa bouteille
nettement carrée et son petit coup de coude en fin de gorgée, il est pas mal
quand même. Et pour le prix, pour le prix\ldots} Durant des heures ça se
tortille, ça fait de la grosse poudre, ça s'ostine sur la prochaine toune, tout
ça entrecoupé de petites conversations sur les fauteuils rouges amples mais
quand même angulaires joliment installés en ménage à trois sur le bord de la
rampe,ou l'on peut entendre  avec, évidemment, la vue majestueuse sur la deuxième
moitié en hauteur de la bay window, cette lumière colorée à travers les
échancrures des grands luminaires abstrait de glissants d'étincelles. Les
petites heures approchent et je me retourne à contempler la vie et la jeune
femme avocate sincère et spirituelle qui me fait face dans la cuisine entre le
fridge et le comptoir auquel elle est indolemment accotée et j'voudrais lui
contempler les bas-fonds de l'âme et m'y plonger mais en lieu de réelle
connexion astralement charnelle je plonge avec un air vide ma main gauche dans
un gros bol de cheetos et pendant qu'elle élabore sur la constitutionnalité
rétrocessionnaire; je me liche un à un, lentement, chaque doigt de la main gauche.
La droite est encore dans le bol de cheetos.
Peut-être qu'elle est trop absorbée
dans une masturbation égotique abstraite à travers des mots avec trop de
syllabes mais la jeune avocate empathique ne remarque pas le déclic apathique
qui me traverse les narines.\\

Un rien-calissage zen et méditatif dans le vrai
sens du terme où je fixe un ustensile, n'écoute rien, ni ce qu'elle dit ni les
percées de paroles du bruit de fond constant ni les paroles du rapper
\textit{Lil-Mickey-Royce} 
et après m'être liché les doigts emmène quelques regards autour de moi pour
constater que David et Julie ont quitté Jean est beaucoup trop partit pour me
remonter et Joe est probablement déjà rentré avec quelqu'un(e), je m'avance le 
grand, très grand verre de vin à la main en boit un grand trait et le dépose
sur une corniche car la fenêtre est ouverte et donne sur un faux balcon.
Je descend les escaliers, il faut mouvement, il faut symétrie, spirale 
l'ascenseur est trop uni-dimensionnel.
Une fois sur le trottoir de la grande rue McGill avec ses nouveaux lampadaires
chics et sa belle asphalte large et ondulée et les commerces de luxe et le
post-modernisme en action je me dirige vers le large. Arrivé à la promenade
sous-jacente à la piste cyclable je m'avance vers la fin d'un pier, comme une 
presqu'île pittoresque. Je m'allonge tranquillement et verse la main à l'eau.
Elle est tiède surprenamment. Je me relève donc et me défait lentement de mes
vêtements. D'abord chaussettes puis pantalons puis chemises, je garde le caleçon parce
que je n'ai aucune pulsion d'indécence, je respecte les normes qu'elles soit 
régressivement puritaine ou non.\\

Et HOP un bon plongeon revigorant. 
J'entame une large brasse, de longs mouvements tout en douceur et commence
à naviguer les berges du Saint-Laurent.
J'aime la brasse car on peut pour quelques instants respirer et bien regarder
devant et peut-être même échanger quelques mots et yeux doux avant de se
ré submerger.\\

\hfill\noindent --\textit{ Mais voyons qu'est-ce que vous faites Monsieur}

\vspace{1em}
--Je n'en aucune idée mon cher mais une chose est
sure vous vous m'emmer-blblblb

\end{document}
