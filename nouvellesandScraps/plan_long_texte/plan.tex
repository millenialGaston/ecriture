\documentclass{article}

\usepackage{comment}
\usepackage[english]{isodate}
\usepackage{graphicx}
\usepackage{siunitx}
\usepackage{paracol}
\usepackage[utf8]{inputenc}
\usepackage{csquotes}
\usepackage{setspace}
\usepackage[bookmarks=true]{hyperref}
\usepackage{bookmark}
\usepackage{pdfpages}
%\includepdf[pages={1}]{myfile.pdf}
\renewcommand{\baselinestretch}{1.5}
%%For definitions:
\usepackage{amsthm}
\theoremstyle{plain}
\newtheorem{thm}{Theorem}[section] % reset theorem numbering for each chapter
\theoremstyle{definition}
\newtheorem{defn}[thm]{Definition}

\usepackage[margin=1in]{geometry}
\setlength{\parindent}{0pt}






\begin{document}

La présente a pour but d'expliquer mon intention de poursuivre des études
en création littéraire aux cycles supérieures. Mes études au 
Bac ont été principalement en sciences pures (génie électrique et mathématiques
pures) je puisse donc concevoir que mon parcours ait l'air atypique dans le 
contexte de cette application. En fait les mathématiques et la littérature sont
mes deux principaux intérêts depuis très longtemps. Le choix quelques années 
auparavant s'est fait sur plusieurs facteurs, l'argent, évidemment, mais 
aussi un certain malaise face à la création en institution. Malgré tout
j'ai toujours continué à lire énormément, de la poésie et de la fiction, 
et j'ai eu la chance de suivre un cours de création littéraire, donné
par Perrine Leblanc à ma dernière session à McGill. Ce dernier m'a démontré
l'importance du groupe dans la création. Malgré le fait que l'écriture
est une activité principalement solitaire, l'échange et le partage
me paraissent maintenant cruciaux dans le processus. C'est pourquoi
je veux maintenant poursuivre mes études en création à l'UQAM, pour 
sortir de mon cubicule qui semble se comprimer autour de moi plus le temps
passe. J'ai une idée de projet de création longue, ne disons pas roman,
car ce serait un peu téméraire. Le texte que je présente pour mon application
est en fait deux extraits de ce projet sur lequel j'ai essayé de plancher 
cet été malgré mon travail à temps plein. J'en avait fait un plan d'écriture
pour Perrine il y a de cela quelque mois, je le présente ci-dessous:\\



Je prévois un squelette qui me permettrait
de m'attarder ou de peaufiner certaines scènes, certains évènements au gré de la
richesse que je semble en déployer. Une série de fragments, d'environ une demie
page, me semble non seulement une structure flexible dans mon approche mais
aussi intéressante en tant que telle. \\

J'ai lu NW de Zadie Smith récemment où la relation entre deux femmes du
Nord-Ouest de Londres est racontée en série de fragments numérotés. On est dans
l'ellipse, en quelques dizaines de pages les années déferlent, mais on a droit
à une écriture parfois à la fois extrêmement précise et compacte.  C'est une
forme qui m'attire beaucoup. Des morceaux denses liés de façon souple par
quelques lignes vides, le collage, l'assemblage. D'un autre coté Stoner de John
Williams m'a beaucoup marqué. C'est un récit d'environ 220 pages qui comprime à
son essence la vie d'un professeur de littérature dans le sud des états unis au
début du 20 siècle. Encore une fois le rythme est frappant, la narration peut
parcourir des mois à chaque phrases et par la suite s'attarder à une scène avec
une extrême précision. Cette capacité à varier le rythme de la narration
m'attire beaucoup, pas seulement du point de vue technique mais aussi comme
philosophie littéraire, un temps hétérogène dont les mouvements subtils sont la
colonne vertébrale du récit.\\

Ceci étant dit, je me laisse une possibilité, en travaillant un des bouts de
récits il se peut que l'un finisse par se dessiner plus clairement à mon
esprit, que le procédé semble devenir artificiel. Alors je travaillerais plus
en profondeur la matière plus prometteuse, développerais le fragment et les
autres pourront soit être recyclés, jetés, ou réintégrés par une digression du
narrateur.\\ 

\end{document}
