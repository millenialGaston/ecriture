\documentclass{article}

\usepackage{comment}
\usepackage[english]{isodate}
\usepackage{graphicx}
\usepackage{siunitx}
\usepackage{paracol}
\usepackage[utf8]{inputenc}
\usepackage{csquotes}
\usepackage{setspace}
\usepackage[bookmarks=true]{hyperref}
\usepackage{bookmark}
\usepackage{pdfpages}
%\includepdf[pages={1}]{myfile.pdf}
\renewcommand{\baselinestretch}{1.5}
%%For definitions:
\usepackage{amsthm}
\theoremstyle{plain}
\newtheorem{thm}{Theorem}[section] % reset theorem numbering for each chapter
\theoremstyle{definition}
\newtheorem{defn}[thm]{Definition}

\usepackage[margin=1in]{geometry}
\setlength{\parindent}{0pt}





 \begin{document} Je prévois un squelette qui me permettrait
de m'attarder ou de paufiner certaines scènes, certains évènements au gré de la
richesse que je semble en déployer. Une série de fragments, d'environ une demie
page, me semble non seulement une structure flexible dans mon approche mais
aussi intéressante en tant que telle. \\

J'ai lu NW de Zadie Smith récemment où la relation entre deux femmes du
Nord-Ouest de Londres est racontée en série de fragments numérotés. On est dans
l'ellipse, en quelques dizaines de pages les années déferlent, mais on a droit
à une écriture parfois à la fois extrêmement précise et compacte.  C'est une
forme qui m'attire beaucoup. Des morceaux denses liés de façon souple par
quelques lignes vides, le collage, l'assemblage. D'un autre coté Stoner de John
Williams m'a beaucoup marqué. C'est un récit d'environ 220 pages qui comprime à
son essence la vie d'un professeur de littérature dans le sud des états unis au
début du 20 siècle. Encore une fois le rythme est frappant, la narration peut
parcourir des mois à chaque phrases et par la suite s'attarder à une scène avec
une extrème précision. Cette capacité à varier le rythme de la narration
m'attire beaucoup, pas seulement du point de vue technique mais aussi comme
philosophie littéraire, un temps hétérogène dont les mouvements subtils sont la
colonne vertébrale du récit.\\

Ceci étant dit, je me laisse une possibilité, en travaillant un des bouts de
récits il se peut que l'un finisse par se dessiner plus clairement à mon
esprit, que le procédé semble devenir artificiel. Alors je travaillerais plus
en profondeur la matière plus prometteuse, developperais le fragment et les
autres pourront soit être recyclés, jetés, ou réintégrés par une disgressions du
narrateur.\\ 

Je refais brièvement le point sur mon personnage, c'est au moment que je
vois comme le présent, le temps central, une femme d'environ 32 ans; Jolie
Beausécourt. Elle est de taille moyenne, c'est à dire autour de 5 pieds quatre, les
cheveux cuivré, plutôt mince, les yeux fins en amande, les traits délicats, le
corps souvent caché sous des vêtements amples, le teint hâlé, le rire bref en
éclats surprenants. Elle est originaire de Saint-Georges en Beauce, jouait du
folk à l'enfance et s'est trempée dans le punk à l'adolescence sans jamais devenir 
fanatique de styles musicaux et s'y coller la personnalité, toujours éclectique
dans ses gouts. Au départ; initiée pas sa mère aux classiques hippie des années 60.  Elle
a par la suite découvert le jazz, la musique africaine et indienne, a fait des
études de niveau universitaire dans la perfomance jazz à la guitare à McGill.
Jolie est plutôt sobre à l'ordinaire, modérée mais peut se laisser aller de
temps à autre. Auquel cas on peut voir un côté juvénile et attendrissant de sa
personnalité, une douceur, une ouverture gênée au monde, des rires plus
cristallins s'en émane et il est alors facile de la faire rougir avec un
commentaire taquin ou un compliment passé rapidement. Elle a tendance à
machouiller ses cheveux ou le cordon du capuchon du kangoroo gris-bleu qu'elle
porte à outrance et dans lequel elle aime se dissimuler, aime aussi se cacher derrière ses
minces cheveux. Lorsqu'elle joue elle ne s'avance jamais, pas même pour ses
solos souvent virtuose mais qui savent puiser des forces dans le silence. \\ 

\end{document}
