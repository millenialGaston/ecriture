\documentclass{article}

\usepackage{comment}
\usepackage[english]{isodate}
\usepackage{graphicx}
\usepackage{siunitx}
\usepackage{paracol}
\usepackage[utf8]{inputenc}
\usepackage{csquotes}
\usepackage{setspace}
\usepackage[bookmarks=true]{hyperref}
\usepackage{bookmark}
\usepackage{pdfpages}
%\includepdf[pages={1}]{myfile.pdf}
\renewcommand{\baselinestretch}{1.5}
%%For definitions:
\usepackage{amsthm}
\theoremstyle{plain}
\newtheorem{thm}{Theorem}[section] % reset theorem numbering for each chapter
\theoremstyle{definition}
\newtheorem{defn}[thm]{Definition}

\usepackage[margin=1in]{geometry}
\setlength{\parindent}{0pt}






\begin{document}
La poussière tombe un peu tranquillement et je commence à comprendre le coup de
tête, la décision brusque.  Tu te sentais coincée, à bout de souffle. Je sais
que les dernières semaines ont été épuisantes. Si moi j'ai pris du retard
j'imagine que c'était un peu catastrophique de ton côté.  C'est quelque chose
que j'avais sentit un peu, la raison pour laquelle je te demandais pourquoi est
ce que tu voulais me poser des questions comme quel est mon rapport à
sexualité. J'imagine que ce peut être casual parfois, dans ce contexte on avait
pas mal passé le casual.\\

Ce qui me fait de la peine c'est que tu t'es sentit pris au piège avec moi au
point que la seule issue était de me ramener mes choses dans un petit sac,
brièvement, comme ça, la technique band-aid ou boulet de cannon dépendemment du
point de vue. Pour foncer au travers du mur. Que tu te sentes à ce point
découragée et prisonnière que d'en parler avec moi n'était pas une option.\\

Je sais que l'on a des trucs qui réagissent assez fortement entre nous deux,
question communication et personnalité. 

Par contre je pense que ce qui enflammait le tout était plus au final
l'impatience de connaître l'autre, l'impatience face à un amour qui est
clairement vrai, c'est pas toujours si clair, quand on le voit on se presse, on
cours un peu trop vite, et on trébuche, souvent.

De toutes les conversations que l'on a eu je pense que c'était probablement la
plus pertinente que l'on aurait du avoir; comment gérer. Comment modérer, se
laisser de l'espace pour souffler, pour travailler, pour les amis. Au lieu
d'imploser ensemble, fusionnellement. C'était la conversation a avoir, le seul
véritable problème à régler, se laisser respirer.

Je croyais que c'était possible, de décider de prendre parfois une semaine ou
on se voyait moins, pour avoir du temps pour soi, pour souffler. Peut-être que
dans d'autres relations tu pouvais partager ton quotidien sans cesse. Se
réveiller avec l'autre souvent, la majorité du temps. Clairement ça ne
s'appliquait pas à nous à ce moment.

C'était trop intense, la petite flamme. Et on s'est fait mal, un peu, je pense
pas que c'était profond commme blessures, c'était des accrochages, ça faisait
des flamèches oui, ça sonnait comme une explosion mais c'était juste des
pétards, ça sonnait fort et on s'en remettait vite, juste de plus en plus
fatigués.  Pas fatigués de sommeil ou de sport, fatigués d'émotion.

Je pense que l'on aurait pu trouver moyen de se reposer.

Et il faut que je parle de Vendredi:
Vendredi, ce que j'essayais de dire, c'est que ce genre d'intensité émotive
comme on en avait eu cette semaine, les cris suivit des pleurs, pour moi ce
n'est pas anodin, ca sonne comme un symptome de quelquechose pour moi. je
voulais pas te pathologiser mais pas banaliser non plus,
j'essayais de savoir comment pressentir ces moments, s'arrêter avant que ça
saute. \\ J'aurais du juste me la fermer.
\newpage
Peut-être ces choses sont plus parlables pour moi, et dire ce
genre de choses est moins épouvantable, parce que j'en parle beaucoup, je
normalise, je pense que tout le monde a des bibittes, je les cherche pas,
vraiment pas, mais quand je pense en voir ou en deviner  j'ai de la misère à
ignorer, et encore plus à communiquer ce que j'en pense:

que c'est un petit problème qui fait parfois très mal mais ce gère très bien si
il y a confiance. Si tu penses pas que j'essaye de discréditer tes émotions.

Mais au final ça a sonné commme "tes fuckée caro, ta
personnalité est maladive", ce qui me fait \emph{bad trip}.  Parce que bien sur ta
personnalité pour moi est absolument pas une maladie.
Ta personnalité me donne envie de chanter des chansons de disney...
\hfill pi je hais disney.


Quand je vois quelqu'un qui a de la peine qui
explose par moment et bien je veux aider, sans être un psy, juste un copain, qui
aide un peu, au mieux qu'il peut, sans relation de pouvoir, sans controller,
juste être là, si l'autre le veut bien. C'était tout, au final.

Et le pire c'est que j'étais probablement complètement dans le champ.
Et que je t'ai blessé et achevé notre relation d'un coup.
Parce que je pouvais pas dismiss mon intuition, mon doute, mon inquiétude,
ni la partager doucement.

Je pensais que l'on pourrait en rejaser éventuellement, s'ajuster, 
se reposer. Prendre des marches au lieu de conversations interminables
la nuit. Je pensais tout ça jusqu'à ce matin, même maintenant je continue
à le penser un peu.

Mais une rupture, un break up, ça fesse plus que tout autre conversation
intense ou chicane. Je sais que j'étais en train de me bruler, de manquer des
bouts à l'école, dans les applications de job que j'étais supposer faire etc. 
que c'était le cas pour toi aussi.
mais il me semblait que tout ça pouvait se régler.

Et c'est pour ça que ta réaction ma faché autant. D'un coup, la claque du break
up, ça fait vasciller, et c'est dur de s'en remmettre.  Maintenant si on se
revoit dans 2 semaines il y aura une distance étrange, un pont brulé.

On dit Break up c'est pas pour rien, ça casse quelque chose.  Et ça me rend
triste parce que je t'aime comme je sais pas le dire.  Et que je pensais qu'on
pourrait se donner une chance, de l'espace pour respirer, penser à des
techniques, grandir et maturer un peu ensemble, autant à travers nos bibittes
que nos moments d'euphorie, de joie paisible.

Je sais pas si c'est encore possible, j'ai pas envie d'essayer de te convaincre
de changer d'idée parce que le problème était là.  Je trouve ca dommage que ça
a été la solution pour toi, de me dire bye un dimanche matin, pendant que ton lift
t'attends. 

Donc voila, je trouve ça triste, on s'est bruler en quelques semaines, avec de
l'amour et du stress. J'ose croire que l'on aurait pu se gérer, se modérer, mais là j'ai de
la peine, tu as de la peine, je sais que tu n'est pas froide.  Et je parle au
passé et au conditionnel. Et ça m'enrage.  Hier j'avais l'impression que tout
ce dont on avait besoin c'était d'un chill pill, d'un peu plus de temps pour
soi, de se calmer, de moins se parler au téléphone et de se voir plus souvent
dehors, prendre des marches.  Et là je sais pas quoi faire.  J'ai l'impression
que tout est finit, comme ça d'un coup.  C'est probablement le cas.  Et ça fait mal

Et ça reste, même si on en reparle, dans une ou deux
semaines, après que "poussière retombe" ce sera étrange et légèrement froid.
C'est toujours un peu surréel revoir un ex, après que la poussière retombe,
pour un debriefing. 
Et en attendant, je peux travailler, faire mes trucs, lire plus.

Je sais quoi faire, mais quoi penser de tout ça aucune idée.

Tout ce que je sais:\\
La peine était là, la joie aussi, l'amour. \\
Et moi qui a fuck up\\
Et toi qui est pu capable.

\clearpage

8 heures plus tard (4am), plus de poussiere.(jetais faché)\\

je sais que j'ai été décalissant, mon commentaire sur l'âge, essaie d'être
au dessus de 8 ans; c'était rough, blessant, pour rien. \\

Mais maintenant plus j'y pense plus la poussière tombe, plus je suis faché
et froid. Et je hais ce sentiment. Ca va surement revirer debord et je vais 
me lever plus triste plus décaliss mais pour l'instant c'est ce qui ce passe.\\

Parce que après tout j'ai l'impression que c'est vraiment pas fair. J'ai été
la pour toi, 2am chez moi ou en apres midi pendant mes cours à venir te rejoindre
à la job. J'ai essayé de te sécuriser, j'ai essayer de modérer; de freiner
les discussions intenses mais en même temps ça avait l'air d'être sécurisant;
un lose-lose situation. Je me sens culpabiliser dès que le filtre
baisse, je laisse sortir un commentaire wack et ça te permets de me 
dire que je te rabaisse. Alors que j'ai passé les derniers mois à tout
faire en fonction de toi. \\

Ma vie s'est organisée autour de toi. Me faire blamer ton anxiété et
ta peine je l'accepte pas. Me faire dropper en 2sc je l'accepte pas. 
C'est unidirectionnel ça, je sais que ça l'a empietté sur ton travail
mais c'est pas une raison. Une rupture c'est pas une réponse magique,
en tout cas je sais que moi je vais être encore brulé et fatigué pour 
un bon bout.\\

Toutes ces peures, ces insécurités, tu m'as fait penser que ça venait
de nous. C'est clair que non. Il y a des choses qui te font peur, il 
y a des choses qui te font mal et c'est pas de ma faute.\\

Je t'aime caliss on a été fusionnels pendant des mois,
l'intensité je suis capable de rentrer dedans mais je l'exige pas.
Tu voulais l'intensité, la passion, la fusion.
Et ensuite tu me dis que c'est too much? Juste quand je suis 
tombé le plus en amour possible avec toi.\\


C'est tes choix de vouloir des relations aussi intenses, des discussions
à n'en pu finir parce que ça te rassure de savoir le plus possible,
bien sur que ça fatigue, que ça l'égratigne, c'est normal, c'est chercher
le trouble si on est pas capable de handle. Moi j'ai aucun problème
à être casual, je te l'ai dit et redit. Je peux t'aimer et te voir 5 fois
semaines, je peux t'aimer et te voir une fois semaine. Mais peut-importe
ce que je fait je me retourne à me faire blamer, et 5min plus tard
adorer, bisouter comme si j'étais un ange, qui s'amusait à te faire 
de la peine.\\

Tu sais on pourrait juste prendre ça relax, arrêter d'être aussi sérieux,
de se sentir aussi obligés d'être constamment en contact. De ne pas s'attendre
à ce que l'autre réponde chaque jour à chaque petit message. \\

et j'y repense, quand est ce que c'est à mon tour caliss de me faire
sécuriser. 

\newpage

apres la conversation

C'est ben trop intense tout ça, ça fait plusieurs fois que ça
l'arrive, on devient trop fusionnels et on veut revenir un peu
dans le temps et etre plus relax.\\

C'est ce que j'ai essayé de faire en disant que cette semaine
on avait beaucoup de travail et que on se verrait moins mais
que c'est pas un éloignement, c'est un peu de repos.\\


J'ai mes torts. Je sais que je suis décalissant parfois, bête, j'ai projeté 
mon anxiété qui me venait de Florence vendredi et c'était
tout simplement stupide.\\

Comme toi aussi tu as projeté des trucs sur moi, c'est normal,
on a un bagage, une expérience, des points de repères.\\

Et je pense qu'un équilibre c'est possible. Moins de drama
c'est possible, c'est trop dramatique tout ça. On a vécu 
2 ans de couples en 2 ans. Je sais pas, je pense que ça se modère,
d'habitude je suis plutot casual, distant et indépendant,
et je pensais que ce serait une bonne idée de jaser en à plus
finir, pour sécuriser. Et je nous ai entrainé dans une spiralle
de bad trip suivit d'épiphanies, on va se le dire, une épiphanie
c'est bien, mais on en a eu bennnn trop\\

Mais c'était cave, on est juste deux personnes qui aime passer du temps
ensemble. Et oui desfois c'est un peu essouflant les relations.\\


Une vrai relation c'est pas un refuge, c'est pas un nid en dehors du
stress de la vie. Ça en fait partie. Desfois on devient gaga et
attendrit et c'est cool, mais c'est pas possible de maintenir ça,
de façon honnête.\\

En rétrospective c'est sur que l'on allait sentir quon se pile dessus,
à parler autant tout le temps, à être aussi analytique, à aller dans
toutes ces zones tendues, qu,est ce que la sexualité, couple ouvert,
honnetêté, futur etc. Je dis pas que c'est toi, j'ai pris le devant
bien comme il faut à un certain moment. Je pensais que ça aidait.\\

Au final c'était plus décalissant qu'autre chose. Moi non plus je veux
pas être brulé de même. Je suis fatigué, à boute. Mais j'ose croire que
l'on est dans le même team. Que c'est pas toxique au point qu'il faille
couper les ponts pour se protéger. On est quand même pas rendu là.
Qu'on puisse en jaser comme cet été quand tu venais me voir et qu'on
jasait, comme en amis. \\

On va pas s'expliquer notre personne l'un à l'autre en 2 mois,
mais on en a déjà fait un esti de bout. Looking back c'est quand même
impressionant, à quel point on rentré dans le monde de l'autre rapidement,
dans sa tête. C'est de l'énergie. On pourrait pas juste  profiter de
cet investissement et prendre des vacances. Encore une fois, le mot
qui me vient en tête c'est impatience, de nous deux bien sur.//

Et c'est pour ça que ça m'a autant décalissé la rupture de même,
je pensais qu'on était à un stade ou on pourrait juste se dire 
que bon, c'est un peu too much tout ça, let s take it easy.
Je comprends pas pourquoi ça devrait passer par un break  up sec,
par un framework. \\

Parce que je sais que t'es à boute, moi aussi, si tu me dis
que t'as besoin de plus de temps dans ta vie, moins de drame,
je te suis à 100\% . Je comprends pas la binarité, couple fusionnel
ou adieux déchirant. Ce "si je te parle chez toi on va revenir ensemble".
Est ce que on peut arrêter de voir ça de façon aussi binaire.\\

C'est pas ou on se voit et on se brule et se détruit ou on coupe les
ponts, je suis pas d'accord, comme si il y avait un grand danger immédiat
et qu'il fallait courrir le plus vite possible. Si tu m'avais dit j'ai besoin
d'un peu de temps je suis à boute, j'aurais tellement comprit.\\

Je comprendrais encore, ce avec quoi j'ai de la misère c'est les grandes
décisions, les résolutions. \\

On aime passé du temps ensemble mais ces deniers temps c'était décalissants,
pour multiples raisons, c'est quoi les options, le pour et le contre de chacune;
c'est pas un point de départ de discussion qui se peut?\\

Comme dans ton analogie du gateau, les di'etes extremes ca marche pas,
faut juste se retenir un peu desfois meme si cest fuking taste parce quen
suite on est gros et on a juste envie de faire une sieste.




\end{document}
