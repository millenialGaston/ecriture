\documentclass{article}

\usepackage{comment}
\usepackage[english]{isodate}
\usepackage{graphicx}
\usepackage{siunitx}
\usepackage{paracol}
\usepackage[utf8]{inputenc}
\usepackage{csquotes}
\usepackage{setspace}
\usepackage[bookmarks=true]{hyperref}
\usepackage{bookmark}
\usepackage{pdfpages}
%\includepdf[pages={1}]{myfile.pdf}
\renewcommand{\baselinestretch}{1.5}
%%For definitions:
\usepackage{amsthm}
\theoremstyle{plain}
\newtheorem{thm}{Theorem}[section] % reset theorem numbering for each chapter
\theoremstyle{definition}
\newtheorem{defn}[thm]{Definition}

\usepackage[margin=1in]{geometry}
\setlength{\parindent}{0pt}








\usepackage[utf8]{inputenc}
\usepackage{lmodern}
 \renewcommand{\familydefault}{\sfdefault} 
\begin{document}
\textbf{Un truck}

Sa dernière et sixième année à Montréal elle se trouve au même nid. Trois
colloques, toutes gentilles, le grille pain est efficace, il y a une petite
gallerie en avant avec un set de patio éclectique, des bordées de coussins et
des chaises adirondaques. C’est le début de l’été elle s’assoit sur l’un des
fauteils, fait ses lectures en après-midi. Elle a ammenné avec elle dehors
quelques volumes de poésie et des revues type national-geographic avec des
grandes photos de mammifères marins immenses et paisbles et des chutes d’eau
tropicales comme si c’était le monde dans lequel on vivait. La rue Casgrain lui
fait face et elle prend une pause une heure ou deux après s'être
réveillée, boit un café et fait du people watching en mangeant une courge
spaguetti. Un bol de salade au couscous traine quelque part, le soleil ne devrait
pas tarder à s'éteindre.



Depuis quatre ou cinq mois c’est Cédric, plus jeune de quelques années, il est
mignon et gentil quelque peu naïf et anxieux mais il séduit avec ses yeux
nuageux d’ailleurs un peu loin. Il débarque de son vélo lui glisse un sourire
s'assied a terre lui demande de raconter sa journée il reste de la lumière
ils en profitent pour en faire de l'ellipse le temps ça se caresse ça se
domestique, on lui donne des commandes avec des biscuits les minutes grésillent
comme un bruit blanc le ciel bleu gris délavé comme un vieux jeans. La salle de bain est 
à repeindre juste les bobettes à remettre il en met partout. La pizza est à terre Jolie
aussi, assise en lotus la bierre aux lèvres. 


Il est un peu pathétique il lui laisse des poèmes écrit en coin de tables 
sur le bord du matelas au sol lorsqu'il se lève pour aller chez lui et elle
dort un peu encore, c'est la sièste, ce soir elle chante dans un bar. Ça la
touche malgré tout; elle en garde quelques un par la suite, au cours du reste
de sa vie d’adulte, ils la suivent
dans une petite boîte en carton, par exemple :
\begin{flushright}
Avec tes taches de rousseur, poussières de feu
ça éclate tu es mon camion d’aube tu
verse dans le large une greffe de rayons
jette les murs pour des clairières
l’herbe haute l’air sec m’exfolie
le creu du sourire
‘ s’ouvre et on se berce hier s’arrête
demain commence après on verra
peut être
à petits pas
dort sans moi t’es bien
tu t-loves un peu dans les draps
\end{flushright}

Sa dernière et sixième année à Montréal elle se trouve au même nid. Trois
colloques, toutes gentilles, le grille pain est efficace, il y a une petite
gallerie en avant avec un set de patio éclectique, des bordées de coussins et
des chaises adirondaques. C’est le début de l’été elle s’assoit sur l’un des
fauteils, fait ses lectures en après-midi. Elle a ammenné avec elle dehors
quelques volumes de poésie et des revues type national-geographic avec des
grandes photos de mammifères marins immenses et paisbles et des chutes d’eau
tropicales comme si c’était le monde dans lequel on vivait. La rue Casgrain lui
fait face et elle prend une pause une heure ou deux après s'être
réveillée, boit un café et fait du people watching en mangeant une courge
spaguetti. Un bol de salade au couscous traine quelque part, le soleil ne devrait
pas tarder à s'éteindre.



Depuis quatre ou cinq mois c’est Cédric, plus jeune de quelques années, il est
mignon et gentil quelque peu naïf et anxieux mais il séduit avec ses yeux
nuageux d’ailleurs un peu loin. Il débarque de son vélo lui glisse un sourire
s'assied a terre lui demande de raconter sa journée il reste de la lumière
ils en profitent pour en faire de l'ellipse le temps ça se caresse ça se
domestique, on lui donne des commandes avec des biscuits les minutes grésillent
comme un bruit blanc le ciel bleu gris délavé comme un vieux jeans. La salle de bain est 
à repeindre juste les bobettes à remettre il en met partout. La pizza est à terre Jolie
aussi, assise en lotus la bierre aux lèvres. 


Il est un peu pathétique il lui laisse des poèmes écrit en coin de tables 
sur le bord du matelas au sol lorsqu'il se lève pour aller chez lui et elle
dort un peu encore, c'est la sièste, ce soir elle chante dans un bar. Ça la
touche malgré tout; elle en garde quelques un par la suite, au cours du reste
de sa vie d’adulte, ils la suivent
dans une petite boîte en carton, par exemple :
\begin{flushright}
Avec tes taches de rousseur, poussières de feu
ça éclate tu es mon camion d’aube tu
verse dans le large une greffe de rayons
jette les murs pour des clairières
l’herbe haute l’air sec m’exfolie
le creu du sourire
‘ s’ouvre et on se berce hier s’arrête
demain commence après on verra
peut être
à petits pas
dort sans moi t’es bien
tu t-loves un peu dans les draps
\end{flushright}
\newpage
\section{dump}

d’une journée sans fin, ça s’étire j’en ai le cafard
d’être de même, comme avard de paix de mieux
je me sens bien c’est l’éloge de pas grand chose, même rien
parce que c’est pas grandiose, juste cohérent
J’ai envie de te crier des bols de lentilles
faut que ça cesse que je retrouve mon vide
sans lui je sais plus ;
parce que dans le fond pourquoi
rassasié d’extase vite que je vous trouve une discorde
j’hallucine l’écrin je le sais
le vrai se condense pas sur des brillants de douceur
Il faut que les vents fauchent de la scrape
l’ammène dans les airs il faut des noyaux
pour que ça condense, un grain de sel
une tache de poussière
\begin{comment}
Les apartements aussi se succèdent aussi, quelques mésaventures, des
chinchillas des champignons un coloc un peu creepy mais sur l’ensemble c’est
sain c’est frais. Elle prend de la maturité comme un vent pur d’automne qui
pique le nez un peu. 
\end{comment}
\end{document}
